

\documentclass[letterpaper,11pt]{article}

\usepackage[english, russian]{babel}
\usepackage[T2A]{fontenc}
\usepackage[utf8]{inputenc}
\usepackage{soulutf8}

\usepackage{latexsym}
\usepackage[empty]{fullpage}
\usepackage{titlesec}
\usepackage{marvosym}
\usepackage[usenames,dvipsnames]{color}
\usepackage{verbatim}
\usepackage{enumitem}
\usepackage[hidelinks]{hyperref}
\usepackage{fancyhdr}
\usepackage[english]{babel}
\usepackage{tabularx}
\input{glyphtounicode}


%----------FONT OPTIONS----------
% sans-serif
% \usepackage[sfdefault]{FiraSans}
% \usepackage[sfdefault]{roboto}
% \usepackage[sfdefault]{noto-sans}
% \usepackage[default]{sourcesanspro}

% serif
% \usepackage{CormorantGaramond}
% \usepackage{charter}


\pagestyle{fancy}
\fancyhf{} % clear all header and footer fields
\fancyfoot{}
\renewcommand{\headrulewidth}{0pt}
\renewcommand{\footrulewidth}{0pt}

% Adjust margins
\addtolength{\oddsidemargin}{-0.5in}
\addtolength{\evensidemargin}{-0.5in}
\addtolength{\textwidth}{1in}
\addtolength{\topmargin}{-.5in}
\addtolength{\textheight}{1.0in}

\urlstyle{same}

\raggedbottom
\raggedright
\setlength{\tabcolsep}{0in}

% Sections formatting
\titleformat{\section}{
  \vspace{-4pt}\scshape\raggedright\large
}{}{0em}{}[\color{black}\titlerule \vspace{-5pt}]

% Ensure that generate pdf is machine readable/ATS parsable
\pdfgentounicode=1

%-------------------------
% Custom commands
\newcommand{\resumeItem}[1]{
  \item\small{
    {#1 \vspace{-2pt}}
  }
}

\newcommand{\resumeSubheading}[4]{
  \item
    \begin{tabular*}{0.97\textwidth}[t]{l@{\extracolsep{\fill}}r}
      \textbf{#1} & #2 \\
      \textit{\small#3} & \textit{\small #4} \\
    \end{tabular*}\vspace{-7pt}
}

\newcommand{\resumeSubheadingThree}[3]{
    \item
    \begin{tabular*}{0.97\textwidth}{l@{\extracolsep{\fill}}r}
      \textbf{#1} & \textit{\small #2} \\
      \textit{\small #3} \\
    \end{tabular*}\vspace{-7pt}
}

\newcommand{\resumeProjectHeading}[2]{
    \item
    \begin{tabular*}{0.97\textwidth}{l@{\extracolsep{\fill}}r}
      \small#1 & \textit{\small #2} \\
    \end{tabular*}\vspace{-7pt}
}

\newcommand{\resumeProjectDescriptionHeading}[3]{
    \item
    \begin{tabular*}{0.97\textwidth}{l@{\extracolsep{\fill}}r}
      \small#1 & \textit{\small #2} \\
      \textit{\small#3} \\
    \end{tabular*}\vspace{-7pt}
}

\newcommand{\resumeSubItem}[1]{\resumeItem{#1}\vspace{-4pt}}

\renewcommand\labelitemii{$\vcenter{\hbox{\tiny$\bullet$}}$}

\newcommand{\resumeSubHeadingListStart}{\begin{itemize}[leftmargin=0.15in, label={}]}
\newcommand{\resumeSubHeadingListEnd}{\end{itemize}}
\newcommand{\resumeItemListStart}{\begin{itemize}}
\newcommand{\resumeItemListEnd}{\end{itemize}\vspace{-5pt}}



%-------------------------------------------
%%%%%%  RESUME STARTS HERE  %%%%%%%%%%%%%%%%%%%%%%%%%%%%


\begin{document}

%----------HEADING----------
% \begin{tabular*}{\textwidth}{l@{\extracolsep{\fill}}r}
%   \textbf{\href{http://sourabhbajaj.com/}{\Large Sourabh Bajaj}} & Email : \href{mailto:sourabh@sourabhbajaj.com}{sourabh@sourabhbajaj.com}\\
%   \href{http://sourabhbajaj.com/}{http://www.sourabhbajaj.com} & Mobile : +1-123-456-7890 \\
% \end{tabular*}

\begin{center}
    \textbf{\Huge \scshape Мельников Кирилл} \\ \vspace{1pt}
    \textbf{Backend-developer} \\
    \small +7-913-078-5055 $|$ 
    \href{mailto:x@x.com}{\underline{kdmelnikov@edu.hse.ru}} $|$ 
    \href{https://t.me/kmelnikovmh}{\underline{t.me/kmelnikovmh}} $|$
    \href{https://github.com/kmelnikovmh}{\underline{github.com/kmelnikovmh}}
\end{center}


%-----------EDUCATION-----------
\section{Образование}
  \resumeSubHeadingListStart
    \resumeSubheading
      {НИУ ВШЭ}{Санкт-Петербург}
      {Бакалаврская программа «Прикладная математика и информатика», «ПМИ»}{2024 -- н.в.}
    \resumeSubheading
      {Инженерно-математическая школа ВК}{Дистанционно}
      {Продвинутый C++ и C, Слушатель}{сен 2024 -- янв 2025}
  \resumeSubHeadingListEnd


%-----------EXPERIENCE-----------
\section{Опыт Работы}
  \resumeSubHeadingListStart
  \vspace{-4pt}
    \resumeSubheadingThree
      {Частный преподаватель математики}{2022 -- 2025}
      {Профиру, 31 отзыв, оценка 4,97 }{}
%      \resumeItemListStart
%        \resumeItem{Developed a REST API using FastAPI and PostgreSQL to store data from learning management systems}
%        \resumeItem{Developed a full-stack web application using Flask, React, PostgreSQL and Docker to analyze GitHub data}
%        \resumeItem{Explored ways to visualize GitHub collaboration in a classroom setting}
%      \resumeItemListEnd

    \resumeSubheadingThree{Администратор спота}{2021 -- 2022}
    {Сёрф школа <<Maverick>>}{}
      
% -----------Multiple Positions Heading-----------
%    \resumeSubSubheading
%     {Software Engineer I}{Oct 2014 - Sep 2016}
%     \resumeItemListStart
%        \resumeItem{Apache Beam}
%          {Apache Beam is a unified model for defining both batch and streaming data-parallel processing pipelines}
%     \resumeItemListEnd
%    \resumeSubHeadingListEnd
%-------------------------------------------

  \resumeSubHeadingListEnd


%-----------PROJECTS-----------
\section{Проекты C, C++}

\begin{itemize}[leftmargin=0in, label={}]
  \item\small{
    {Протестировано на разных системах, компиляторах (GCC, Clang, MSVC, Apple Clang) \vspace{-6pt}}. Используется CMake
  }
  \item\small{
    {Написаны тесты (doctest, bash-testing).  \vspace{-2pt}}
  }
\end{itemize}

    \resumeSubHeadingListStart
      \resumeProjectDescriptionHeading
          {\href{https://github.com/kmelnikovmh/KuMyS-Artifact-Manager}{\underline{\textbf{Artifact-Manager}}} $|$ \emph{CppRestSDK, Folly, FastCGI, MongoDB, Nginx, Docker Compose}}{фев 2025 -- н.в.}
          {Менеджер репозиториев для корпоративных сетей, аналог Nexus Sonatype, командный проект}
          \resumeItemListStart
            \resumeItem{Разработал асинхронный мультимодульный сервер для обработки apt-запросов}
            \resumeItem{Настроил корректный перехват запросов от apt на стороне клиента}
          \resumeItemListEnd

      \resumeProjectHeading
          {\textbf{Bank-Server} $|$ \emph{Boost.Asio}}{мар 2025}
          \resumeItemListStart
            \resumeItem{Разработал многопоточный банковский сервер с независимой обработкой TCP-клиентов}
            \resumeItem{Реализовал потокобезопасные классы со сторогой гарантией исключений}
          \resumeItemListEnd

      \resumeProjectHeading
          {\textbf{C-Calculator} $|$ \emph{C}}{апр 2025}
          \resumeItemListStart
            \resumeItem{Написал на Си арифметическую библиотеку (double) и линейный парсер с функциями и обработкой ошибок}
          \resumeItemListEnd

      \resumeProjectHeading
          {\textbf{Bmp-Image} $|$ \emph{C}}{апр 2025}
          \resumeItemListStart
            \resumeItem{Реализовал класс со строгой гарантией для побайтовой работы с изображениями: сrop, rotate за $O(\texttt{ответа})$}
          \resumeItemListEnd

      \resumeProjectHeading
          {\textbf{My-Test} }{фев 2025}
          \resumeItemListStart
            \resumeItem{Разработал header-only библиотеку для юнит-тестирования с макросами \texttt{CHECK, TEST\_CASE, SUBCASE, ...}}
          \resumeItemListEnd

      \resumeProjectHeading
          {\textbf{Vector} }{май 2025}
          \resumeItemListStart
            \resumeItem{todo}
          \resumeItemListEnd

      \resumeProjectHeading
          {\textbf{Ptrs}  }{дек 2024}
          \resumeItemListStart
            \resumeItem{Реализовал потокобезопасные \texttt{unique\_ptr<T, Deleter>, shared\_ptr<T>} с поддержкой incomplete types}
          \resumeItemListEnd

      \resumeProjectHeading
          {\textbf{Interpreter assembler-like}  }{фев 2025}
          \resumeItemListStart
            \resumeItem{Спроектировал матричный интерпретатор со строгой гарантией исключений и арифметикой матриц}
          \resumeItemListEnd

      \resumeProjectHeading
          {\textbf{Tic-Tac-Toe} $|$ \emph{Boost.Dll}}{ноя 2024}
          \resumeItemListStart
            \resumeItem{Разработал консольную игру в крестики-нолики 10x10 с динамической поддержкой плагинов отображения}
          \resumeItemListEnd

      \resumeProjectHeading
          {\textbf{Остальные проекты доступны в репозитории Cpp-Labs (Widgets-Position, Bigint)} }{}

    \resumeSubHeadingListEnd

\section{Проекты Python}
    \resumeSubHeadingListStart

      \resumeProjectHeading
          {\textbf{Split-Check} $|$ \emph{Django, Bootstrap}}{мар 2025}
          \resumeItemListStart
            \resumeItem{Реализовал веб-приложение с: авторизацией, созданием чеков с изображениями, их просмотром и удалением}
          \resumeItemListEnd

          
    \resumeSubHeadingListEnd



%
%-----------PROGRAMMING SKILLS-----------
\section{Технические Навыки}
 \begin{itemize}[leftmargin=0.15in, label={}]
    \small{\item{
     \textbf{Языки}{: C++, C, NoSQL (MongoDB) $|$ Python, SQL (MySQL), HTML/CSS $|$ Nginx $|$  Bash} \\
     \textbf{Библиотеки}{: Boost, CppRestSDK, Folly, FastCGI, Doctest} \\
     \textbf{Инструменты разработки}{: Git, CMake, Docker, Docker Compose, Colima, VS Code} \\
     \textbf{Математика}{: Матанализ, Общая алгебра, Линейная алгебра, Дискретная математика, Матлогика} \\
     \textbf{Алгоритмы}{: ... много. Наиболее запомнившиеся: MergeSort, QuakeHeap, Кнут, ZPP, 2-SAT, DSU, k-OPT}
    }}
 \end{itemize}



%-------------------------------------------
\end{document}